\begin{abstract}
\hrule
\medskip
The modelling and simulation of the human physiological system has largely attracted the attention of the scientific community in the past decade. With rapid and steady increase in computing power and advances in image acquisition technology, creating patient-specific, image based models of several spatial and time scales have been made possible. Various international projects have put together research groups across the world interested in this subject. In Brazil, the most notable example is that of the MACC project (Medicina Assistida por Computa\c{c}\~a Cient\'ifica), led by the HeMoLab group of the LNCC (Laborat\'orio Nacional de Computa\c{c}\~ao Cient\'ifica). Among other objectives, the MACC project aims at studying and developing models of the human cardiovascular system, including the simulation of surgical procedures. 

One application of models of the cardiovascular system is in the treatment of atherosclerosis using stents. Correct stent choice and deployment is paramount for a successful treatment and this task is much dependent on the expertise of the specialist in charge. In the first part of this project proposal, a solution is presented for the automatic assessment of vascular stenoses and aneurysms and automatic choice of stent parameters (type, diameter, length, deployment location) based on the estimation of healthy blood vessels with statistical shape models and on the simulation of blood flow after the stent has been deployed. This solution is inspired by the author's previous experience with the automatic assessment and stenting of tracheal stenosis and the expertise of the HeMoLab group in simulations of blood flow. 

The second part of this project proposal presents an extension of the MACC project itself, by building an image based gas exchange model to simulate lung function. This extension, herein referred to as The PulMoLab, will be another building block in the development of the Brazilian Virtual Physiological Human project, an effort that has been started by the members of the MACC project.

This project proposal thus describes the many challenges to be faced in the execution of the two parts and the approaches to solve them. It also shows how the project shall be integrated into the work currently done at the HeMoLab/LNCC, how it will benefit from their expertise, and how it will contribute to their work. Finally, tentative schedules for the execution of the project and plans for the use of funding are presented.

%  As propostas deverão ser apresentadas na forma de projeto de pesquisa. Recomenda-se que este
% projeto apresente as seguintes informações, de forma a permitir sua adequada análise por parte do Comitê
% Julgador:
% \begin{itemize}
%  \item i. resumo do projeto de pesquisa proposto, incluindo objetivos e metas a serem cumpridos, com os
% respectivos indicadores de desempenho;
% \item ii. cronograma de execução do projeto;
% \item iii. orçamento detalhado, especificando a aplicação do auxílio à pesquisa do projeto;
% \item iv. grau de interesse e comprometimento de empresas ou instituições com o escopo da proposta,
% quando for o caso;
% \item v. descrição das atividades a serem desenvolvidas pelos demais participantes do projeto, em
% especial pelo beneficiário da cota adicional de bolsa;
% \item vi. disponibilidade efetiva de infra-estrutura e de apoio técnico para o desenvolvimento do projeto e;
% \item vii. previsão dos ganhos e benefícios para a instituição no país com a vinda do bolsista Atração de
% Jovens Talentos;
% \end{itemize}

\medskip
\hrule
\end{abstract}



