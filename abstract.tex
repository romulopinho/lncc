\begin{abstract}
\hrule
\medskip
Understanding 

Atherosclerosis figures as one of the leading cardiovascular diseases, which remain as the major cause of death in developed countries. Image analysis and processing is an important tool in the diagnosis and treatment of vascular diseases. Minimally invasive interventions are increasingly becoming the treatment option of choice, given the reduced surgical risk for the patient. A good example is the use of stents to reopen vascular stenosis or to protect the dilated walls of aneurysms from further expansion. Stent choice remains, nevertheless, a subjective decision. An inadequate stent may migrate to another location, may exert exaggerated pressure on the vessel walls, may not properly cover the affected area, etc. 

This document describes a project proposal whose aims are twofold.  for the automatic detection and quantification of stenosis and aneurysms and for the automatic prediction of optimal patient-specific stent parameters. The proposed methods build upon the work presented in my PhD thesis, which focussed on the assessment and stenting of tracheal stenosis. They also integrate physical properties of blood vessels and stents, as well as stent deployment and blood flow simulations, into the stent choice process. 
\medskip
\hrule
\end{abstract}

As propostas deverão ser apresentadas na forma de projeto de pesquisa. Recomenda-se que este
projeto apresente as seguintes informações, de forma a permitir sua adequada análise por parte do Comitê
Julgador:
i. resumo do projeto de pesquisa proposto, incluindo objetivos e metas a serem cumpridos, com os
respectivos indicadores de desempenho;
ii. cronograma de execução do projeto;
iii. orçamento detalhado, especificando a aplicação do auxílio à pesquisa do projeto;
iv. grau de interesse e comprometimento de empresas ou instituições com o escopo da proposta,
quando for o caso;
v. descrição das atividades a serem desenvolvidas pelos demais participantes do projeto, em
especial pelo beneficiário da cota adicional de bolsa;
vi. disponibilidade efetiva de infra-estrutura e de apoio técnico para o desenvolvimento do projeto e;
vii. previsão dos ganhos e benefícios para a instituição no país com a vinda do bolsista Atração de
Jovens Talentos;

