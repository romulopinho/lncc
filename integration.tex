\chapter[Integration in the Laboratory and Collaborations]
{
\hrule
\medskip\medskip\medskip
Integration in the Laboratory and Collaborations
\medskip\medskip\medskip
\hrule
}
\label{chap:integration}

\section{Laboratory}

My intention of joining the HeMoLab group of the LNCC is on the one hand to benefit from their expertise in the several areas correlated to this project proposal. In particular, as mentioned before, the HeMoLab/LNCC is the leading research group of the INCT-MACC project, among the objectives of which is the development of models of blood circulation (this research is under the supervision of Prof. Dr. Raul Feij\'oo -- coordinator of the INCT-MACC -- and Prof. Dr. Pablo Blanco -- coordinator of the HeMoLab associate laboratory of the INCT-MACC). On the other hand, it will be my role to contribute to group's expertise by 1) bringing to the lab my own experience with the automatic assessment and patient-specific stenting of tracheal stenosis and applying it to the cardiovascular domain and 2) building upon the group's existing work on the cardiovascular system to develop the proposed PulMoLab project.

The HeMoLab/LNCC also have at their disposal two internal computer clusters for large scale computations. Their expertise in the parallel computing domain and the processing of massive image data have been demonstrated through their participation in many nationwide projects, such as the INCT-MACC itself. This cluster is part of a larger grid structure comprising several research instutions across the country, the SINAPAD\footnote{https://www.lncc.br/sinapad}, of which the LNCC is the coordinating instution. Such expertise and tools will be invaluable for the simulation parts of the proposed project. 

Most importantly, the HeMoLab/LNCC also profit from a very multidisciplinary nature and strong links with the clinical practice, especially that of the {\it Instituto do Cora\c{c}\~ao} (InCor), in S\~ao Paulo. This will potentially give me the opportunity to evaluate the results of the proposed project with real patient data and to have immediate feedback from physicians. 

\section{Collaborations}

During my PhD in Belgium, I established a very good relationship with Prof. Dr. Jan Sijbers, my thesis supervisor, in the VisionLab group of the University of Antwerp. The group has extensive experience with medical image segmentation problems and modelling of cylindrical objects \citep{Huysmans1}, which could be interesting in the context of this project. I would like to bring the VisionLab and the HeMoLab/LNCC together during the course of the proposed project, so that they can benefit from each other's work.

My stay in Belgium also allowed me to get acquainted with the scientific staff of the FluidDA company \citep{DeBacker2010, DeBacker2008} and the work of the Stent Research Unit of the University of Ghent \citep{deBeule}. The former are specialized in CFD simulations of airflow to study particle deposit and breathing problems. The latter is focussed on the design of vascular stents using numerical simulations. As such, they represent potential collaborators for the proposed project or at least a valuable source of information.  

I also envision collaborations with the academic and industrial partners of the European Project THROMBUS, whose aim is to evaluate the efficacy of the use of stents in neurological aneurysms, and researchers involved with the state of the art in gas exchange modelling and simulations. Likewise, I intend to be in close contact with the research groups involved in the Lung Physiome project, possibly the leading researchers in the field. All these groups combine expertise in different areas and will certainly be a good source of information for this project.

Specifically for Part II of the proposed project, The PulMoLab, an interaction with the group I currently work at the Centre L\'eon B\'erard/CREATIS, headed by Dr. David Sarrut, can be rather beneficial. The referred group has large experience in lung segmentation problems and image based simulations of lung function in the context of radiotherapy treatments. 

\chapter[Schedule and Use of Funding]
{
\hrule
\medskip\medskip\medskip
Schedule and Use of Funding
\medskip\medskip\medskip
\hrule
}
\label{chap:schedule}
\section{Tentative Schedules}

Due to relocation issues, I intend to start working on this project in March, 2013, which is the limit start date of this call. Having this plan in mind, below are two tables with tentative schedules for the each part of this project proposal, knowing that both parts will be carried out in parallel, on a 50-50 time share basis.

\paragraph{Part I}
The objective in the short term, roughly the first 6 months, is to try to directly apply the ASM used for the trachea to cases of vascular stenosis and aneurysms. There are reasons to believe that this step tends to be rather straightforward, requiring only few modifications to the original method, if any. The possibly biggest challenge would be the segmentation and surface modelling of the vessels, which, in the first instance, could be simplified (e.g., not taking bifurcations into account) and accomplished with existing techniques so as to yield acceptable results in a short period of time. 

The following step, to be carried out during the next 18 months, will be the extension of the ASM with anatomical information about the vascular tree. Bifurcations will also be taken into account. 

Finally, the remaining 12 months will be dedicated to the stent choice and simulation parts. The starting point of this step will be the stent choice with a generic stent and vessel model. In other words, stents' and vessels' physical properties will not be taken into account. In this way, we will be able to at least evaluate if the generic stents computed with the ASM are adequate. Later, blood flow and stent deployment simulation results will be used to improve the statistical model and, ultimately, the automatic stent choice.  

\begin{table}[h]\centering
\begin{tabular}{c|c|c|c|c|c|c|}
\cline{2-7}
 & \multicolumn{2}{|c|}{Y1} & \multicolumn{2}{|c|}{Y2} & \multicolumn{2}{|c|}{Y3} \\ \cline{2-7}
 & S1 & S2 & S3 & S4 & S5 & S6 \\ \hline
\multicolumn{1}{|c|}{Apply existing model} & \cellcolor{green} & & & & & \\ \hline
\multicolumn{1}{|c|}{Segmentation and modelling} & & \cellcolor{green} & \cellcolor{green} & \cellcolor{green} & & \\ \hline
\multicolumn{1}{|c|}{Stent choice and simulations} & & & & & \cellcolor{green} & \cellcolor{green} \\ \hline
\end{tabular}
\caption{Part I -- Tentative schedule for 3 years, subdivided into semesters.}
\label{tab:schedule1}
\end{table}

\paragraph{Part II}

The core of this part, i.e., the Airflow and Gas Exchange Simulations, is to be carried out during the first two years of the project. This will give us enough time to completely understand the problem, properly study the literature, identify opportunities for innovation, and implement them. 

The next two tasks will be carried out in parallel with the first one (one in each year), since they are somewhat independent and approaches to solve the related challenges are already defined. The two tasks will also be carried out during the first two years of the project and could be, for example, topics for Master theses. 

The last step, to be carried out during the third year, depends on the former three, and must therefore wait until all the rest has been mastered and roughly developed. 

\begin{table}[h]\centering
\begin{tabular}{c|c|c|c|c|c|c|}
\cline{2-7}
 & \multicolumn{2}{|c|}{Y1} & \multicolumn{2}{|c|}{Y2} & \multicolumn{2}{|c|}{Y3} \\ \cline{2-7}
 & S1 & S2 & S3 & S4 & S5 & S6 \\ \hline
\multicolumn{1}{|c|}{Airflow and Gas Exchange Simulations} & \cellcolor{green} & \cellcolor{green} & \cellcolor{green} & \cellcolor{green} & & \\ \hline
\multicolumn{1}{|c|}{Segmentation of Airway Tree} & \cellcolor{green} & \cellcolor{green} & & & & \\ \hline
\multicolumn{1}{|c|}{Segmentation of Pulmonary Vascular Tree} & &  & \cellcolor{green} & \cellcolor{green} & & \\ \hline
\multicolumn{1}{|c|}{Correlations Between Pulmonary Structure and Function} & & & & & \cellcolor{green} & \cellcolor{green} \\ \hline
\end{tabular}
\caption{Part II -- Tentative schedule for 3 years, subdivided into semesters.}
\label{tab:schedule2}
\end{table}


\section{Use of Funding}

Under the general supervision of Professor-Doctors Raul Feij\'oo and Pablo Blanco, I will be the coordinator and main researcher of the proposed project, and therefore the direct benefiter of the offered grant. However, I acknowledge that the project is rather ambitious and that I will not be able to work on it alone. Naturally, I will be interacting with the research and technical staff of the INCT-MACC-HeMoLab and of the LNCC in a whole. Still, the project will largely benefit from the extra {\em Inicia\c{c}\~ao Cient\'ifica} or {\em Inicia\c{c}\~ao Tecnol\'ogica Industrial} scholarships, which are also part of this call. The scholarship holder will have the responsibility of implementing some of the algorithms devised during the project and/or those from literature. Moreover, many of the challenges described above may be solved in the context of Master or even PhD theses, funding for which may be available from the usual Brazilian funding bodies.

Part of the extra research funding (R\$ 20.000,00/year) will thus be employed in the purchase of equipment for myself and for the selected scholarship holder. Occasionally, part of this funding may also be used for trips to collect data, in case they cannot be transferred electronically.
