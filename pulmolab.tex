\chapter[Part II: The PulMoLab]
{
\hrule
\medskip\medskip\medskip
Part II: The PulmoLab
\medskip\medskip\medskip
\hrule
}
\label{chap:pulmolab}

\section[Context]{Context}

The need for a quantitative understanding of in vivo lung function leads to the need for reliable model representation of the in vivo lung structure. The lung primarily comprises parenchymal (gas exchange) tissue, airways, arteries, and veins. The treelike structures of the airways and blood vessels are intimately bound to—and suspended within—the parenchyma (alveolar tissue). Function in the lung is strongly influenced by the spatial positioning of the airways and vessels \citep{TawhaiM2011,Tawhai2010}, calling for a patient-specif approach in simulating and analysing lung structure and function. 

With advances in modern medical imaging and the steady and rapid increase in computing power, it is now possible to derive models that replicate the many geometric features of individual lungs, and to simulate function within these geometric models, taking into account their different spatial and time scales \citep{Tawhai2010,Tawhai2000,Tawhai08,Burrowes2005}.

A number of international projects have put together groups interested in the modelling and simulation of lung function, among them the Human Lung Atlas \citep{Hoffman2004,Li2003}, the Lung Physiome \citep{TawhaiM2011}, and the AIRPROM-VPHNoE\footnote{http://cordis.europa.eu/projects/rcn/97980_en.html}. In Brazil, to the best of my knowledge, such an effort has not yet been started. However, human physiological modelling and simulation has been a topic of interest of the HeMoLab group 



Here, we review progress in constructing multiscale finite element models of lung structure and function that are aimed at providing a computational framework for bridging the spatial scales from molecular to whole organ. These include structural models of the intact lung, embedded models of the pulmonary airways that couple to model lung tissue, and models of the pulmonary vasculature that account for distinct structural differences at the extra- and intra-acinar levels. Biophysically based functional models for tissue deformation, pulmonary blood flow, and airway bronchoconstriction are also described. The development of these advanced multiscale models has led to a better understanding of complex physiological mechanisms that govern regional lung perfusion and emergent heterogeneity during bronchoconstriction

In \citep{TawhaiM2011}

The main interest here is thus to extend the work developed by the HeMoLab group in the context of the MACC project \citep{Blanco2012,Blanco2010,Blanco2009a,Blanco2007,Golbert2012,Malossi2011,Urquiza2006}, adding to it the modelling of the gas exchange process. The primary objective is to investigate the existing literature on the subject, find the common points with the work developed in the HeMoLab, and carry out the integration between the existing models of vascular system and the new models of gas exchange. Some challenges in this process are rather clear, which are show below along with the approaches to solve them. 

\section{Segmentation of the Airway Tree}

\challenge

It has been shown that CFD simulations of airflow are largely improved when patient-specific image based models of the airways are employed \citep{Tawhai2010,Vial2005}. In effect, the segmentation of the airway tree has been a topic of study for quite some time, but the automatic segmentation remains an open problem. Achieving correct manual (or semi-automatic) segmentation can be painstaking, non-reproducible, and prone to error. Yet, state-of-art, fully automatic algorithms still fall short of achieving the same results obtained with manual interventions \citep{Lo}. The complexity lies in the little image resolution at the lower levels of the airway tree. Noise and other artefacts such as partial volume effect make the automatic identification of smaller branches even more difficult.

\approach

One algorithm for the automatic segmentation of airway trees \citep{Pinho:Airways2} implements an iterative region growing with adaptive, cylindrical ROIs that employs anatomical information in the detection and elimination of leaks, a common problem in region growing. The method also includes an initialization step to automatically detect the starting point of the trachea, which is given as a seed point to the region growing algorithm. 

The intention is to further extend the referred method and try to incorporate new ideas from the litterature \citep{Lo} in order to increase the branch detection count and to reduce the number of segmentation leaks and false positves. For example, tube enhancement filters \citep{ORLO-09} may help in at least isolating the tubular structures in the image, from where the detection process may be triggered. Branch generation algorithms \citep{Tawhai2000} will also be investigated.

\section{Segmentation of the Pulmonary Vascular Tree}

\challenge

The vessels that compose the pulmanary vascular tree have a many-to-one relationship with the airway bronchi. Namely, each bronchial airway is related to one arterial vessel, but there are more arteries than branches of the airways. The same applies to the venous system. These extra vessels, referred to as "supernumerary", together with the other vessels are responsible for transporting blood to the gas exchange surface, within the lung parenchyma. Correct modelling of the vessel tree is thus an important step in trying to perform gas exchange simulations. Therefore, it is necessary to correctly segment the vascular tree in order to obtain valid patient-specif geometric models to be used in the CFD simulations. 

\approach

In theroy, the algorithms proposed for the segmentation of the airways can be employed in the segmentation of pulmonary vascular tree, for the two trees have, overall, the same strucutre \citep{TawhaiM2011}. Therefore, we will investigate the use of the algorithms described in \citep{Lo}, for the airways, and also those proposed exclusively for the segmentation of the vascular tree, such as \citep{Dongen,Ebrahimdoost,Gutierrez,Linguraru,Shikata,Wala}.


\section{Airflow and Gas Exchange Simulations}

\challenge

An appropriate multiscale model will capture the important features of function at each spatial or temporal scale while maintaining as much computational simplicity as possible. Physical forces acting on the surface of the intact lung are transmitted to the level of the gas exchange tissue (where they hold open the small airways and blood vessels) and on down to the level of cells and molecules where stress modulates local function.

On the one hand, this is probably the most challenging part of the PulMoLab proposal. On the other hand, it is also the one that tends to benefit the most from the expertise of the MACC/HeMoLab/LNCC in the modelling and simulations of the human vascular system. 

\approach

\section{Correlations Between Pulmonary Structure and Function}

\challenge

As stated before, one the most significant benefits of airflow and gas exchange simulations is to establish correlations between lung structure and function. In other words, the aim to answer questions such as how the shape of the airways influences turbulence of the airflow or particle deposition, or how the geometry of the lungs and its deformations upon breathing affect the overall gas exchange process in the normal patient or in the presence of diseases. 

\approach

Since the aim is to find correlations in which shape variations are involved, the idea is to employ the aforementioned Active Shape Models (ASMs) \citep{Cootes} for this task. To the best of my knowledge, ASMs have never been used to correlate lung shape and function. Furthermore, the reader is reminded that research on the {\em statistical analysis of tree-like shapes}, e.g., the airway tree, is still in its early stages \citep{Feragen2011,Feragen2012}, so there may be many opportunities for innovation is this case. This approach will benefit from the results obtained with the estimation of healthy arteries, described in Section \ref{sec:healthyvessels}, and vice-versa.

