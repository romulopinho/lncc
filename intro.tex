\chapter[Introduction]{
\hrule
\medskip\medskip\medskip
Introduction
\medskip\medskip\medskip
\hrule
}
\label{chap:introduction}

The advance of scientific computing has fostered the development of systems to model complex biological phenomena. Possibly chief among these is the simulation of the physiology of the human body, comprising the modelling of multiple spatial (from 1nm for the pore size of ion channels to 1m for the size of the human body) and temporal scales (from 10$^{-6}$s for the Brownian motion to 10$^{9}$s of our life-time), the biochemistry, the biophysics and the cellular anatomy of tissues and organs. These simulations enable both the understanding of organ functioning in normal conditions as well as the prediction of their behaviour under abnormal conditions or medical procedures. In such a way, additional information can be provided to the diagnosis and treatment of a patient and the planning of medical interventions. 

The scientific community worldwide has long been motivated by the understanding of the human physiological system. Several joint efforts and consortia have been established around the world in order to carry on research in this field. Among them, it is worth mentioning the following:  

\begin{itemize}
\item The Virtual Physiological Human Project \\ (http://www.vph-noe.eu/)
\item Physiome Project of the International Union of Physiological Sciences \\ (http://www.physiome.org.nz/).
\item Virtual Human Project \\ (http://www.ornl.gov/sci/virtualhuman/).
\item The Digital Human Project – Federation of American Scientist \\ (http://www.fas.org/dh).
\item EuroPhysiome Project HaeMOdel – Mathematical Modeling of the Cardiovascular System \\ (http://mox.polimi.it/it/progetti/haemodel/?en=en).
\item SimBio – Simulation in Biology \\ (www.simbio.de/).
\item GEMSS – Grid for Medical Simulation Service \\ (http://www.it.neclab.eu/gemss/).
\end{itemize}

In Brazil, the most notable example for the modelling and analysis of human physiological systems is that of the INCT-MACC (Instituto Nacional de Ci\^encia e Tecnologia em Medicina Assistida por Computa\c{c}\~ao Cient\'ifica) project, led by HeMoLab group of the LNCC (Laborat\'orio Nacional de Computa\c{c}\~ao Cient\'ifica). Among the many objectives of the INCT-MACC, we may highlight: 

\begin{itemize}
\item Modelling and simulation of the human cardiovascular system including the simulation of surgical procedures.
\item Modelling and simulation of medical procedures for craniofacial traumatism (including the process of the reconstruction of craniofacial prostheses).
\item Advanced processing of medical images, taking into account visualization and three-dimensional reconstruction of patterns with medical relevance, and its applications in the modelling and computational simulation of physiological systems and image based diagnosis.
% \item Development of collaborative virtual environments for virtual and augmented reality, as well as for telemanipulation in the medicine for training and formation of human resources, and for surgical planning.
% \item System for public health monitoring including remote monitoring and emergency medical attendance (acute myocardial infarction).
% \item Support for multimedia traffic in medical videoconferences.
% \item Cyberenvironments for high performance distributed computing for medical applications in the areas mentioned above.
% \item Human resources formation in the areas mentioned above.
\end{itemize}

The first item above has an interesting application in the treatment of atherosclerosis. Atherosclerosis is a pathology of the blood vessels that occurs when substances such as fat and cholesterol accumulate in the vessel walls, forming plaques. These plaques cause stenoses (vessel narrowing) and aneurysms (weakening and swelling of the vessel wall). Such conditions may lead to severe consequences, for example heart attacks, strokes, and the rupture of the vessel wall, all of them life threatening \citep{Gennest,Libby}.

\begin{sloppypar}
Often, atherosclerosis is handled with stents, but choosing the correct stent parameters (type, length, diameter, and deployment location) is challenging and remains a specialist-dependent and subjective task. Stents are also commonly employed in pulmonary medicine, in cases of airway obstruction \citep{Chin,Freitag1,Freitag2,Freitag3,LeeP,Saito,Venuta}. The problems of finding the correct stent parameters are also very similar to those in vascular stents, e.g., in \citep{Bemmel}. \citet{Pinho:Trachea4} showed that tracheal stent parameters can be automatically computed with the use of Statistical Shape Models \citep{Cootes}. These models are capable of estimating the shape of narrowed tracheae as if they were healthy, from which point the assessment of the stenosis and the calculation of the stent parameters follow naturally. In theory, the same principle could be applied to determining vascular stent parameters. However, the process becomes more complicated due to larger variations in vessel geometry, 
interactions between stents and vessel walls, and desired blood flow characteristics. Modelling and simulation of the cardiovascular system as carried out in the INCT-MACC project could improve the stent choice process. Moreover, post-deployment simulations could optimize stent development even further, aiming for the ideal stent parameters for a specific patient, with a specific medical condition. 
\end{sloppypar}

\begin{sloppypar}
In the first part of the proposed project, the intention is to develop a solution for the problem of automatic quantification of stenosis and stent choice in the treatment of vascular stenoses and aneurysms. To date, such problem has only been solved with semi-automatic methods \citep{Gremse01092011,Scherl200721,HERN-06b,Bemmel}. The goal is then to improve previous solutions with an {\em automatic} process that optimizes the {\em patient-specific} choice of stent type, deployment location, length, and diameter. 
\end{sloppypar}

The second part part of the project, in turn, proposes to extend the INCT-MACC project itself, by building an image based gas exchange model to simulate lung function. Lung physiology and gas exchange process are perfect examples of the multi-spatial and temporal scales involved in the functioning of the human body. Large scale events, such as lung wall and airway deformations upon breathing, and microscopic scale events, such as the transmission of oxygen to blood cells in the lung parenchyma, represent the type of interactions that need to be taken into account when constructing such a simulation model. Despite being a very challenging task, it opens the doors of several research areas, e.g., the segmentation of the lungs, their lobes, and the airway and vascular trees, computer simulations of airflow, the passage of oxygen into blood cells and the release of CO$_2$ out of them, etc. The proposed extension, herein referred to as {\em The PulMoLab}, will be another building block in the development of the 
Brazilian Virtual Physiological Human project, an effort that has been started by the members of the INCT-MACC project \citep{Blanco2010,Blanco2009a,Blanco2012,Urquiza2006,Larrabide2007}. 

The proposed project therefore aims at studying and solving the problems above and evaluating the solutions through simulations and in the clinical domain. In order to achieve it, the project will be developed within the HeMoLab group of the LNCC, whose expertise in the domains of medical image processing and modelling and simulations of the human vascular system will be an invaluable contribution.

The following chapters set forth the main challenges involved in the project and the approaches to solve them. Chapter \ref{chap:stenosis} presents the ideas proposed for the first part of the project, that is, the problem of automatically determining stent parameters for the treatment of vascular stenoses and aneurysms. For the second part, chapter \ref{chap:pulmolab} describes the importance of The PulMoLab and the research areas of which that will be investigated. Chapter \ref{chap:integration} describes how the proposed project will fit into the framework of the HeMoLab/LNCC group, how it will benefit from the group's expertise, and in which ways it will add value to their existing work. Finally, Chapter \ref{chap:schedule} shows tentative schedules for the development of the project, as well as how the project funding will be employed.


