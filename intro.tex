\chapter[Introduction]{
\hrule
\medskip\medskip\medskip
Introduction
\medskip\medskip\medskip
\hrule
}


The advance of scientific computing has fostered the development of systems to model complex biological phenoma. Possibly chief among these is the simulation of the physiology of the human body, comprising the modelling of the several multiple spatial (from 1nm for the pore size in ion channels to 1m for the size of the human body) and temporal scales (from 10-6s for the Brownian motion to 109s of our life-time), the biochemistry, the biophysics and the cellular anatomy, tissues and organs. These simulations enable both the understanding of organ functioning in normal conditions as well as the prediction of their behaviour under abnormal conditions or medical procedures. In such a way, additional information can be provided to the diagnosis and treatment of a patient and the planning of medical interventions. 

The scientific community worldwide has long been motivated by the understanding of the human physiological system. Several joint efforts and consortia have been established around the world in order to carry on research in this field. Among them, it worth mentioning the following:  

\begin{itemize}
\item Physiome Project of the International Union of Physiological Sciences \\ (http://www.physiome.org.nz/).
\item Virtual Human Project \\ (http://www.ornl.gov/sci/virtualhuman/).
\item The Digital Human Project – Federation of American Scientist \\ (http://www.fas.org/dh).
\item EuroPhysiome Project HaeMOdel – Mathematical Modeling of the Cardiovascular System \\ (http://mox.polimi.it/it/progetti/haemodel/?en=en).
\item SimBio – Simulation in Biology \\ (www.simbio.de/).
\item GEMSS – Grid for Medical Simulation Service \\ (http://www.it.neclab.eu/gemss/).
\end{itemize}

In Brazil, the most notable example of the development of systems for the modelling and analysis of human physiological systems is that of the MACC (Medicina Assistida por Computa\'c\~a Cient\'ifica) project, led by the LNCC (Laborat\'orio Nacional de Computa\'c\~a Cient\'ifica). Among the objectives of the MACC, we can emphasize


% promoted a development without precedents in the history of the human society. The popularization of the personal computer, the advent of the internet, the development of wireless communication, of the high performance distributed computing, distributed databases and data mining for knowledge discovery, monitoring techniques, scientific visualization, virtual reality, numerical simulation and computational modeling of complex systems, nowadays permeate all the human activities giving rise to huge and deep changes.
% 
% In medicine, this new reality had its roots in the beginning of the 20th Century, when it was called telemedicine (according to World Health Organization, “telemedicine is the supply of services related to healthcare when distance is a critical factor. Such services are provided by healthcare professionals, using communication and information technologies…”). However, this is currently outdated in front of the possibilities that arise with the use of the newer technologies mentioned above. For example, through the computational modeling of complex physiological systems that couple, by means of several multiple spatial scales (from 1nm for the pore size in ion channels to 1m for the size of the human body) and temporal scales (from 10-6s for the Brownian motion to 109s of our life-time), the biochemistry, the biophysics and the cellular anatomy, tissues and organs, is possible to gain insight into their functioning under normal conditions, as well as under conditions changed by pathological processes or medical procedures, giving additional information that contribute to the enhancement of diagnosis, treatment and planning of diverse medical procedures.
% 
% In this way, the scientific computation produces huge and deep changes in the medicine because it allows:
% 
% \begin{itemize}
% \item a synthesis of the image based diagnosis that, when coupled to the modeling and simulation, permits the development of new therapeutic techniques in real-time to improve medical procedures and treatments;
% \item the development of models and accurate simulators of the different systems within the human body and its inter-relation integrating anatomy, physiology, biomechanical properties, cellular biology and biochemistry for therapeutic and research applications, and also for human resources formation and training;
% \item to develop a “virtual body” for each patient in order to serve as a repository for diagnosis, pathologies and other medical information about the patient. In turn, this “virtual body” allows to increase the communication between patient and physician, furnishing a reference for exams, pathologies and changes that take place as time goes by;
% \item to make use of these models and simulators of high accuracy for surgical planning, and medical training. These simulators permit a real interaction between the user and parts of our body, represented by realistic physical and physiological properties, for educational and research purposes as well as for the development of medical applications.
% \end{itemize}
% 
% 
% Evidences of this new trend can be found in projects developed in the United States and in several countries of the European Community such as:
% 
% \begin{itemize}
% \item Physiome Project of the International Union of Physiological Sciences (http://www.physiome.org.nz/).
% \item Virtual Human Project, (http://www.ornl.gov/sci/virtualhuman/).
% \item The Digital Human Project – Federation of American Scientist (http://www.fas.org/dh).
% \item EuroPhysiome Project HaeMOdel – Mathematical Modeling of the Cardiovascular System http://mox.polimi.it/it/progetti/haemodel/?en=en).
% \item SimBio – Simulation in Biology (www.simbio.de/).
% \item GEMSS – Grid for Medical Simulation Service (http://www.it.neclab.eu/gemss/).
% \end{itemize}



The objectives of the proposed project are twofold: first we aim at developing a solution for the problem of automatic quantification of stenosis and stent choice in the treatment of vascular stenoses and aneurysms. To date, such problem has only been solved with semi-automatic methods \citep{Gremse01092011,Scherl200721,HERN-06b,Bemmel}. The goal is then to improve previous solutions with an {\em automatic} process that optimizes the {\em patient-specific} choice of stent type, length and diameter. The following sections present the main challenges involved in the project and the approaches to solve them. 


